\chapter{系统内存空间分布设计}

\section{系统内存空间}

龙芯 3A 处理器内拥有两个内存控制器 (Memory controler,
或MC),如果能够将地址空间交错分布在两 个内存控制器上,
对于系统的平均访问延迟和平均访问带宽都会带来好处,但是
受到交叉开关上配置窗口个数的限制, 必须采取一定的规则对地址空间分布方法
做出一定的设计。 基于上述考虑,同时为了维护 linux
系统不同内存空间大小的不同需要,并
保证内存空间分布的简单美观,推荐使用下面的规则对内存地址空间进行设计。
当然,根据系统的需要,设计者也可以自定义内存空间分布规则。

\begin{enumerate}
  \item 无论内存大小多大,都必须保证 0x0000\_0000 – 0x0FFF\_FFFF 的低 256MB 空间;
  \item 为了给 IO 设备留出必要的直接访问地址空间 , 0x1000\_0000 – 0x1FFF\_FFFF
    保留不用作空间地址空间;
  \item 1GB 及以上内存空间的剩余部分按照的定义按照下面的公式:
    \begin{eqnarray*}
      Base &=& Size + 0x1000\_0000 \\
      Limit &=& 2*Size - 1,
    \end{eqnarray*}
    其中,Size 是所 有内存的大小,Base 和 Limit 分别是这块空间的基地址和高地址。
\end{enumerate}

举例说明,如果内存大小为 1GB,则内存在系统中的地址空间如下表:
\begin{center}
  \begin{tabular}[h]{|c|c|c|l|} \hline
    & 起始地址 & 结束地址 & \cellalign{c|}{说明} \\ \hhline
    地址0 & 0x0000\_0000\_0000\_0000 & 0x0000\_0000\_0FFF\_FFFF & 0 – 256MB   \\
    地址1 & 0x0000\_0000\_5000\_0000 & 0x0000\_0000\_7FFF\_FFFF & 256MB – 1GB \\ \hline
  \end{tabular}
\end{center}
如果内存大小为 2GB,则内存在系统中的地址空间如下表:
\begin{center}
  \begin{tabular}[h]{|c|c|c|l|} \hline
    & 起始地址 & 结束地址 & \cellalign{c|}{说明} \\ \hhline
    地址0 & 0x0000\_0000\_0000\_0000 & 0x0000\_0000\_0FFF\_FFFF & 0 – 256MB   \\
    地址1 & 0x0000\_0000\_9000\_0000 & 0x0000\_0000\_FFFF\_FFFF & 256MB – 2GB \\ \hline
  \end{tabular}
\end{center}
其他以此类推。

除此之外, 为了使两个内存控制器能够交错使用,我们按照以下方式配置二
级交叉开关上的内存地址空间。

\newcommand{\bmm}[3]{\begin{tabular}{ll} BASE & #1 \\ MASK & #2 \\ MMAP & #3 \end{tabular}}
\begin{longtable}[h]{|p{3cm}|p{4cm}|c|c|}
  \caption{} \label{tab:x2WinConfig} \\
  \hline \cmcolvb{2}{说明} & 窗口号 & 寄存器设置 \\ \hline
  \endhead

  \multicolumn{2}{|p{7cm}|}{使能 BIOS 空间的访问} & 0 &
  \bmm{0x0000\_0000\_1FC0\_0000}{0xFFFF\_FFFF\_FFF0\_0000}{0x0000\_0000\_1FC0\_00F2} \\ \hline
  \multicolumn{2}{|p{7cm}|}{使能 PCI 空间的访问 (仅允许非取指的 UNCACHE 访问通过)} & 1 &
  \bmm{0x0000\_0000\_1000\_0000}{0xFFFF\_FFFF\_F000\_0000}{0x0000\_0000\_1000\_0082} \\ \hline
  使能低 256M 空间的访问 & MC0单通道 256MB 及以上 & 2 &
  \bmm{0x0000\_0000\_0000\_0000}{0xFFFF\_FFFF\_F000\_0000}{0x0000\_0000\_0000\_00F0} \\ \cline{3-4}
                       & 双通道 256MBx2 及以上 (以地址[10]做交错)& 2 &
  \bmm{0x0000\_0000\_0000\_0000}{0xFFFF\_FFFF\_F000\_0400}{0x0000\_0000\_0000\_00F0} \\ \cline{3-4}
                       &                                         & 3 &
  \bmm{0x0000\_0000\_0000\_0400}{0xFFFF\_FFFF\_F000\_0400}{0x0000\_0000\_0000\_00F1} \\ \hline
  使能高地址空间的访问 & MC0 单通道 512M & 4 &
  \bmm{0x0000\_0000\_2000\_0000}{0xFFFF\_FFFF\_F000\_0000}{0x0000\_0000\_1000\_00F0} \\ \cline{2-4}
                       & MC0 单通道 1G   & 4 &
  \bmm{0x0000\_0000\_4000\_0000}{0xFFFF\_FFFF\_C000\_0000}{0x0000\_0000\_0000\_00F0} \\ \cline{2-4}
                       & MC0 单通道 2G   & 4 &
  \bmm{0x0000\_0000\_8000\_0000}{0xFFFF\_FFFF\_8000\_0000}{0x0000\_0000\_0000\_00F0} \\ \cline{2-4}
                       & 双通道256M x 2 (使用地址[10]交错)& 4 &
  \bmm{0x0000\_0000\_2000\_0000}{0xFFFF\_FFFF\_F000\_0400}{0x0000\_0000\_0000\_04F0} \\ \cline{3-4}
                       &                                    & 5 &
  \bmm{0x0000\_0000\_2000\_0400}{0xFFFF\_FFFF\_F000\_0400}{0x0000\_0000\_0000\_04F1} \\ \cline{2-4}
                       & 双通道512M x 2 (使用地址[10]交错)& 4 &
  \bmm{0x0000\_0000\_4000\_0000}{0xFFFF\_FFFF\_E000\_0400}{0x0000\_0000\_0000\_00F0} \\ \cline{3-4}
                       &                                    & 5 &
  \bmm{0x0000\_0000\_4000\_0400}{0xFFFF\_FFFF\_E000\_0400}{0x0000\_0000\_0000\_00F1} \\ \cline{3-4}
                       &                                    & 6 &
  \bmm{0x0000\_0000\_6000\_0000}{0xFFFF\_FFFF\_E000\_0400}{0x0000\_0000\_0000\_04F0} \\ \cline{3-4}
                       &                                    & 7 &
  \bmm{0x0000\_0000\_6000\_0400}{0xFFFF\_FFFF\_E000\_0400}{0x0000\_0000\_0000\_04F1} \\ \cline{2-4}
                       & 双通道1G x 2 (使用地址[10]交错)  & 4 &
  \bmm{0x0000\_0000\_8000\_0000}{0xFFFF\_FFFF\_C000\_0400}{0x0000\_0000\_0000\_00F0} \\ \cline{3-4}
                       &                                    & 5 &
  \bmm{0x0000\_0000\_8000\_0400}{0xFFFF\_FFFF\_C000\_0400}{0x0000\_0000\_0000\_00F1} \\ \cline{3-4}
                       &                                    & 6 &
  \bmm{0x0000\_0000\_C000\_0000}{0xFFFF\_FFFF\_C000\_0400}{0x0000\_0000\_0000\_04F0} \\ \cline{3-4}
                       &                                    & 7 &
  \bmm{0x0000\_0000\_C000\_0400}{0xFFFF\_FFFF\_C000\_0400}{0x0000\_0000\_0000\_04F1} \\ \cline{2-4}
                       & 双通道2G x 2 (使用地址[10]交错)  & 4 &
  \bmm{0x0000\_0001\_0000\_0000}{0xFFFF\_FFFF\_8000\_0400}{0x0000\_0000\_0000\_00F0} \\ \cline{3-4}
                       &                                    & 5 &
  \bmm{0x0000\_0001\_0000\_0400}{0xFFFF\_FFFF\_8000\_0400}{0x0000\_0000\_0000\_00F1} \\ \cline{3-4}
                       &                                    & 6 &
  \bmm{0x0000\_0001\_8000\_0000}{0xFFFF\_FFFF\_8000\_0400}{0x0000\_0000\_0000\_04F0} \\ \cline{3-4}
                       &                                    & 7 &
  \bmm{0x0000\_0001\_8000\_0400}{0xFFFF\_FFFF\_8000\_0400}{0x0000\_0000\_0000\_04F1} \\ \hline
\end{longtable}

\section{系统内存空间与外设 DMA 空间映射关系}

经过这样的配置之后,我们再来介绍内存地址在 DMA 时的使用。传统的 PCI DMA
空间存在于 0x8000\_0000 以上的地址,当设备进行 DMA 操作时, 0x8000\_0000
的访问被映射到 0x0000\_0000 的空间,再与系统内存进行一一的映 射。 在龙芯 3A
中,为了解决大内存的使用 DMA 空间转换的问题,做出如下规 定。

\begin{enumerate}
  \item 系统内存空间 0x0000\_0000 – 0x0FFF\_FFFF 在作为 DMA 空间使用时,
    外设使用 0x8000\_0000 – 0x8FFF\_FFFF 进行访问;
  \item 其它系统内存空间在作为 DMA 空间使用时,总线地址和内存地址一致,
    无需地址转换,直接使用即可。

    以 2GB 内存空间为例,可以得到如下的地址转换表:
    \begin{center}
      \begin{tabular}[h]{|c|c|c|c|c|} \hline
        & \cmcolvb{2}{说明} & 起始地址 & 结束地址 \\ \hhline
        地址0 & 0 – 256MB   & 系统空间 & 0x0000\_0000\_0000 & 0x0000\_0FFF\_FFFF \\ \cline{3-5}
        &             & DMA空间  & 0x0000\_8000\_0000 & 0x0000\_8FFF\_FFFF \\ \hline
        地址1 & 256MB – 2GB & 系统空间 & 0x0000\_9000\_0000 & 0x0000\_FFFF\_FFFF \\ \cline{3-5}
        &             & DMA空间  & 0x0000\_9000\_0000 & 0x0000\_FFFF\_FFFF \\ \hline
      \end{tabular}
    \end{center}
\end{enumerate}

使用 HyperTransport 接口时 ,上面的这种地址转换方法可以通过 HyperTransport
的接收地址窗口配置(FIXME: 参见 XXX 节),使用两组地址窗口来实现:
\begin{center}
  \begin{tabular}[h]{|c|c|c|c|}
    & 说明 & 窗口使能寄存器 & 窗口基址寄存器 \\ \hline
    窗口0 & 0 – 256MB & 0xC000\_0000 & 0x0080\_FFF0 \\ \hline
    窗口1 & 256MB – 2GB & 0xC000\_0080 & 0x0080\_FF80 \\ \hline
  \end{tabular}
\end{center}
这这个设置中,窗口 0 将 0x8000\_0000 – 0x8FFF\_FFFF 的地址转换为 0x0000\_0000 –
0x0FFF\_FFFF 的访问。 窗口 1 将 0x8000\_0000–0xFFFF\_FFFF 的地址转换为
0x8000\_0000–0xFFFF\_FFFF
的访问。由窗口命中的优先级规则,可以得知,0x8000\_0000–0x8FFF\_FFFF
的地址实际上只会被窗口 0 所映射,那么窗口 1 处理的地址实际上为
0x9000\_0000–0xFFFF\_FFFF。

\section{系统内存空间的其它映射方法}

前面两节介绍的系统内存空间映射方法仅是一种有效的参考方式, 系统设计
人员也可以根据自己的需要来重新定义内存映射规则。例如将内存空间全部集中在低地址,而将
IO 地址及系统配置空间映射到较高的空间。

