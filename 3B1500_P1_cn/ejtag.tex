\chapter{EJTAG 调试}

\section{EJTAG 介绍}

EJTAG(Enhanced JTAG)是 MIPS 根据 IEEE1149.1 协议的基本构造和功能扩
展而定义的规范,是一个硬件/软件子系统,用于支持片上调试。EJTAG 规范通
过定义一种新的调试模式,包括专用的指令、运行模式以及地址空间等,与原有 的 MIPS
体系结构很好地融合在一起。 调试模式的基本思想是采用例外处理的机
制,使被调试的软件无法察觉调试器的存在。此外,处理器在调试模式下对地址
空间进行了扩展, 可以访问调试控制寄存器以及映射到内存区域的调试接口。下
图给出了一个常见的 EJTAG 调试系统组成,包括:
\begin{itemize}
  \item 调试主机(Debug Host):运行调试应用程序,控制 EJTAG 线缆
  \item 目标板(Target Board):包含被调试芯片的板卡,提供 EJTAG 接口
\end{itemize}

图 13-1 EJTAG 调试系统 (FIXME)

调试主机可以通过 EJTAG 线使目标板上的处理器进入调试模式,转向执行
调试服务例程。 调试服务的入口有两个, 分别是位于 BIOS 的 0xbfc00480 和位于
EJTAG 调试内存接口的 0xff200200,由调试主机选择。

EJTAG 规范在处理器的调试模式地址空间中划出两块,映射到调试控制寄
存器(drseg)和调试内存接口(dmseg)。当处理器执行调试服务访问调试接口时,
所发出的访存请求将阻塞在调试接口中。此时,调试主机能够检测到调试内存接
口的访存请求,从而对其进行响应。简单的说,调试服务程序可访问位于调试主
机的一段内存空间。

利用调试主机控制的调试用内存空间,调试服务几乎可以完成任意可以想象
的功能,因为调试服务程序的代码本身就可以由调试主机提供 (目前龙芯 3A 和 2G
的样片中不能从 dmseg 取指,但可以执行访存指令)。 龙芯 3A/2G 中还支持 EJTAG
规范的 PC 采样功能。

\section{EJTAG 工具使用}

\subsection{环境准备}

\begin{itemize}
  \item EJTAG 线,并口转 14 针 EJTAG 口
  \item Linux 调试主机,带并口,有 root 权限
  \item PMON 中添加调试服务程序
\end{itemize}

\subsection{PC采样}

在调试主机运行 jtag\_1/4/16,将完成对处理器核的一次 PC 采样。其中程序 后缀是指
EJTAG 接口中处理器核的数目,一个 2G 是 1,一个 3A 是 4,四个 3A 串一起是 16。PC
采样不需要调试服务程序。 3A/2G 中 PC
采样的值不精确,存在抖动,用户需忽略采样结果的低 2 位。 处理器核中真正的 PC
指针可能是所采样 PC 对应指令的程序流后方 0~4 条指令,
即如果包含转移,则有真正的 PC 可能落在转移目标附近。
如果不是有规律的循环或者软件控制的停时钟,PC 采样不应当静止不动。
如果出现,则意味着处理器运行不正常。

\subsection{读写内存}

运行 pracc\_1/4/16,在参数中指定执行调试服务的处理器核号、访问类型、
访问地址、要写的值等。其工作原理是根据所要完成的任务,在调试用内存空间
初始化一段参数区,与调试服务程序相配合,控制调试服务的执行。调试服务会
首先将若干寄存器保存到调试主机,然后基于这些空出的寄存器运行,在读出相
关的参数并执行,最后恢复先前保存过的寄存器。 需注意的是,
这一功能要求执行调试服务的处理器核还能执行指令。如果工
作异常,则可能存在硬件故障。

\subsection{执行说明}

\subsubsection*{读 bfc00480 |值的意义}

\begin{verbatim}
cpu@ubuntu:~/ejtag$ ./pracc_4 0 0xffffffffbfc00480 d r
target = 0, addr = ffffffffbfc00480, dword read
breaking...
ctrl = 00049000
stat = 00008008
pracc write, size=3,  address = 000000000000000f, value = 0000000000000000 |t1
pracc write, size=3,  address = 0000000000000017, value = 0000000000000000 |t2
pracc write, size=3,  address = 000000000000001f, value = fffffffff80e1060 |t3
pracc write, size=3,  address = 0000000000000027, value = ffffffff8000b899 |t4
pracc write, size=3,  address = 000000000000002f, value = ffffffffffffffff |t5
pracc write, size=3,  address = 0000000000000037, value = ffffffff80110000 |t6
pracc write, size=3,  address = 000000000000003f, value = ffffffff800f7428 |t7
pracc write, size=3,  address = 0000000000000047, value = 00000000340000e0 |status
pracc write, size=3,  address = 000000000000004f, value = 0000000080034483 |config
pracc write, size=3,  address = 0000000000000057, value = 0000000040008000 |cause
pracc write, size=3,  address = 000000000000005f, value = 0000000000000033 |a0
pracc write, size=3,  address = 0000000000000067, value = ffffffff80120000 |a1
pracc write, size=3,  address = 000000000000006f, value = 0000000000000000 |a2
pracc read, size=3,  address = 0000000000000207, value = fff3ffffbfc00480
pracc write, size=3,  address = 0000000000000107, value = fff3ffffbfc00480
pracc write, size=3,  address = 0000000000000107, value = 0000000000000003
pracc write, size=3,  address = 0000000000000107, value = 00000000000001fc
pracc write, size=3,  address = 0000000000000107, value = 0000000000000001
pracc write, size=3,  address = 0000000000000107, value = ffffffffbfc00480
pracc write, size=3,  address = 000000000000020f, value = 3c08ff2040a8f800
return 3c08ff2040a8f800                                                        |读返回值
pracc read, size=3,  address = 0000000000000217, value = 0000000000000000
pracc read, size=3,  address = 000000000000021f, value = 0000000000000000
pracc read, size=3,  address = 000000000000000f, value = 0000000000000000
pracc read, size=3,  address = 0000000000000017, value = 0000000000000000
pracc read, size=3,  address = 000000000000001f, value = fffffffff80e1060
pracc read, size=3,  address = 0000000000000027, value = ffffffff8000b899
pracc read, size=3,  address = 000000000000002f, value = ffffffffffffffff
pracc read, size=3,  address = 0000000000000037, value = ffffffff80110000
pracc read, size=3,  address = 000000000000003f, value = ffffffff800f7428
\end{verbatim}

\subsubsection*{写 bfc00480(其实写不进去)}

\begin{verbatim}
cpu@ubuntu:/home/cpu/ejtag$ ./pracc_4 0 0xffffffffbfc00480 d w 0x0
target = 0, addr = ffffffffbfc00480, dword write with 0000000000000000
press <enter> to confirm..
breaking...
ctrl = 00049000
stat = 60008008
pracc write, size=3,  address = 000000000000000f, value = 0000000000000000
pracc write, size=3,  address = 0000000000000017, value = 0000000000000000
pracc write, size=3,  address = 000000000000001f, value = fffffffff80e1060
pracc write, size=3,  address = 0000000000000027, value = ffffffff8000b899
pracc write, size=3,  address = 000000000000002f, value = ffffffffffffffff
pracc write, size=3,  address = 0000000000000037, value = ffffffff80110000
pracc write, size=3,  address = 000000000000003f, value = ffffffff800f7428
pracc write, size=3,  address = 0000000000000047, value = 00000000340000e0
pracc write, size=3,  address = 000000000000004f, value = 0000000080034483
pracc write, size=3,  address = 0000000000000057, value = 0000000040008000
pracc write, size=3,  address = 000000000000005f, value = ffffffff8ec08c00
pracc write, size=3,  address = 0000000000000067, value = ffffffff8ec08400
pracc write, size=3,  address = 000000000000006f, value = 0000000000000000
pracc read, size=3,  address = 0000000000000207, value = 0000000000000000
pracc write, size=3,  address = 0000000000000107, value = 0000000000000000
pracc read, size=3,  address = 0000000000000217, value = fff3ffffbfc00480
pracc read, size=3,  address = 000000000000021f, value = 0000000000000000
pracc write, size=3,  address = 0000000000000107, value = fff3ffffbfc00480
pracc write, size=3,  address = 0000000000000107, value = 0000000000000003
pracc write, size=3,  address = 0000000000000107, value = 00000000000001fc
pracc write, size=3,  address = 0000000000000107, value = 0000000000000001
pracc write, size=3,  address = 0000000000000107, value = ffffffffbfc00480
pracc write, size=3,  address = 000000000000021f, value = 0000000000000000
pracc read, size=3,  address = 000000000000000f, value = 0000000000000000
pracc read, size=3,  address = 0000000000000017, value = 0000000000000000
pracc read, size=3,  address = 000000000000001f, value = fffffffff80e1060
pracc read, size=3,  address = 0000000000000027, value = ffffffff8000b899
pracc read, size=3,  address = 000000000000002f, value = ffffffffffffffff
pracc read, size=3,  address = 0000000000000037, value = ffffffff80110000
pracc read, size=3,  address = 000000000000003f, value = ffffffff800f7428
\end{verbatim}

\subsubsection*{调试服务代码注解}

\begin{lstlisting}
# start.S
/* Debug exception */
        .align  7                       /* bfc00480 */
////////////////////////////////////////////
#define COP_0_DESAVE $31
    .set mips64
    // save context
    dmtc0   t0, COP_0_DESAVE	// 保存一个寄存器用于dmseg指针
    lui     t0, 0xff20	// 
    sd      t1, 0x08(t0)	// 压栈
    sd      t2, 0x10(t0)
    sd      t3, 0x18(t0)
    sd      t4, 0x20(t0)
    sd      t5, 0x28(t0)
    sd      t6, 0x30(t0)
    sd      t7, 0x38(t0)
    dmfc0   t1, COP_0_STATUS_REG	// 输出若干cp0寄存器
    sd      t1, 0x40(t0)
    dmfc0   t1, COP_0_CONFIG
    sd      t1, 0x48(t0)
    dmfc0   t1, COP_0_CAUSE_REG
    sd      t1, 0x50(t0)
    sd      a0, 0x58(t0)	// 其它感兴趣的寄存器
    sd      a1, 0x60(t0)
    sd      a2, 0x68(t0)

#define _t1      9
#define _t2      10
#define _t3      11
#define _t4      12
#define dextu(dest, src, msbd, dlsb) \
    .word ((0x1f<<26)|((src&0x1f)<<21)|((dest&0x1f)<<16)|((( msbd)&0x1f)<<11)|(((dlsb)&0x1f)<<6)|(0x2))
#define dinsu(dest, src, dmsb, dlsb) \
    .word ((0x1f<<26)|((src&0x1f)<<21)|((dest&0x1f)<<16)|(((dmsb)&0x1f)<<11)|(((dlsb)&0x1f)<<6)|(0x6))


    // exec_main
    ld      t1, 0x200(t0)    	// 读参数 addr/size/count
    beqz    t1, read_end
    sd      t1, 0x100(t0)    	// debug...
    dextu   (_t2, _t1, 2-1, 48-32)
    sd      t2, 0x100(t0)    	// debug...
    dextu   (_t3, _t1, 9-1, 50-32)
    sd      t3, 0x100(t0)    	// debug...
    dextu   (_t4, _t1, 1-1, 47-32)
    sd      t4, 0x100(t0)    	// debug...
    dsubu   t4, $0, t4      	// sign bit extend
    dinsu   (_t1, _t4, 58-32, 48-32) // address back
    sd      t1, 0x100(t0)    	// debug...
    // case t2 0,1,2,3  -> lb,lh,lw,ld
    beqzl   t2, 1f
    lb      t5, 0x0(t1)
    addiu   t2, t2, -1

    beqzl   t2, 1f
    lh      t5, 0x0(t1)
    addiu   t2, t2, -1

    beqzl   t2, 1f
    lw      t5, 0x0(t1)
    addiu   t2, t2, -1

    ld      t5, 0x0(t1)
1:
    sd      t5, 0x208(t0)    	// 读返回值
read_end:
    // write
    ld      t1, 0x210(t0)    	// 写参数 addr/size/count
    beqz    t1, write_end
    ld      t5, 0x218(t0)    	//写参数
    sd      t1, 0x100(t0)    	// debug...
    dextu   (_t2, _t1, 2-1, 48-32)
    sd      t2, 0x100(t0)    	// debug...
    dextu   (_t3, _t1, 9-1, 50-32)
    sd      t3, 0x100(t0)    	// debug...
    dextu   (_t4, _t1, 1-1, 47-32)
    sd      t4, 0x100(t0)    	// debug...
    dsubu   t4, $0, t4      	// sign bit extend
    dinsu   (_t1, _t4, 58-32, 48-32) // address back
    sd      t1, 0x100(t0)    	// debug...
    // case t2 0,1,2,3  -> sb,sh,sw,sd
    beqzl   t2, 1f
    sb      t5, 0x0(t1)
    addiu   t2, t2, -1

    beqzl   t2, 1f
    sh      t5, 0x0(t1)
    addiu   t2, t2, -1

    beqzl   t2, 1f
    sw      t5, 0x0(t1)
    addiu   t2, t2, -1

    sd      t5, 0x0(t1)
1:
    sd      t5, 0x218(t0)    	// 写响应
write_end:

    // restore context
    ld      t1, 0x08(t0)	// 退栈
    ld      t2, 0x10(t0)
    ld      t3, 0x18(t0)
    ld      t4, 0x20(t0)
    ld      t5, 0x28(t0)
    ld      t6, 0x30(t0)
    ld      t7, 0x38(t0)
    dmfc0   t0, COP_0_DESAVE
    deret	// 调试例外返回
\end{lstlisting}

\subsection{在线GDB调试}

由于目前样片存在的 bug,在线调试功能需要在常见的 MIPS 调试平台中作
一定量的改动,并增加相应的调试服务程序。该功能尚未实现。

