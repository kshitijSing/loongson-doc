\documentclass[10pt]{standalone}
\usepackage{fontspec}                             % font configuration
\usepackage{xeCJK}                                % allow chinese font
\setCJKmainfont{文泉驿微米黑}                     % default CJK font
\setCJKsansfont{文泉驿点阵正黑}                   % default CJK sans font
\setCJKmonofont{文泉驿等宽微米黑}                 % default CJK mono font
\usepackage{tikz}
\usetikzlibrary{matrix,calc,shapes}

\tikzset{
    base/.style={draw,align=center,inner sep=0,font=\small},
    start/.style = {draw, rectangle, text width=1.9cm},
    decision/.style = {base, diamond,aspect=1.5,
        minimum height=1.8cm, minimum width=2.7cm
    },
    term/.style = {draw, align=center, rectangle, fill=green!60,
        rounded corners,text width=6em, minimum height=1cm
    },
    except/.style={term, fill=pink!50},
    dot/.style = {
        circle,fill=#1,inner sep=0,minimum size=3pt
    }
}

\newcommand{\xy}[1]{\coordinate (#1);}

\begin{document}
\thispagestyle{empty}

\begin{tikzpicture}

% 1. matrix layout
\node[matrix,very thin,column sep=3.3cm,row sep=2.2cm] (matrix) at (0,0) {
    \xy{start} &             &              &          &            & \\
    \xy{VPN}   & \xy{G}      & \xy{V}       & \xy{D}   & \xy{C}     & \\
               & \xy{refill} & \xy{invalid} & \xy{mod} & \xy{noncache} & \xy{cache}\\
};

% 2. put block boxes
\node[start] (start) at (start) {有效虚地址:\\ VPN, ASID};
\node[decision] (VPN) at (VPN) {VPN匹配?};
\node[decision] (G)   at (G)   {$G=1$ or \\ ASID匹配?};
\node[decision] (V)   at (V)   {$V=1$?};
\node[decision] (D)   at (D)   {$D=1$?};
\node[decision] (C)   at (C)   {$C=011_2$?};

\node[except] (refill) at (refill) {XTLB/TLB \\ 重填例外};
\node[except] (invalid) at (invalid) {TLB \\ 无效例外};
\node[except] (mod) at (mod) {TLB \\ MOD例外};

\node[term] (cache) at (cache) {访问Cache};
\node[term] (noncache) at (noncache) {不通过\\ Cache访问};

% 3. arrows
\draw[-latex] (start) -- (VPN);
\draw[-latex] (VPN) -- (G) node[pos=0,right,above] {\tiny 是};
\draw[-latex] (VPN) |- (refill) node[pos=0,left] {\tiny 否};
\draw[-latex] (G) -- (V) node[pos=0,right,above] {\tiny 是};
\draw[-latex] (G) -- (refill) node[pos=0,left] {\tiny 否};
\draw[-latex] (V) -- (D) node[pos=0,right,above] {\tiny 是};
\draw[-latex] (V) -- (invalid) node[pos=0,left] {\tiny 否};
\draw[-latex] (D) -- (C) node[pos=0,right,above] {\tiny 是};
\draw[-latex] (D) -- (mod) node[pos=0,left] {\tiny 否};
\draw[-latex] (C) -| (cache) node[pos=0,right,above] {\tiny 是};
\draw[-latex] (C) -- (noncache) node[pos=0,left] {\tiny 否};
\end{tikzpicture}
\end{document}

